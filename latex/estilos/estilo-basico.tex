% phd-estilo-basico.tex
% aml! 2015-09-08. amaurandi@um.es
% ------------------------------------------------

%\documentclass[a4paper,12pt]{book}
%\documentclass[11pt]{book}
%\usepackage[paperwidth=17cm, paperheight=22.5cm, bottom=2.5cm, right=2.5cm]{geometry}
\usepackage{amssymb,amsmath,amsthm} %paquete para símbolo matemáticos
%\usepackage{apacite} %para usar APA cite en bibligrafia tb modificar
% \usepackage[spanish, es-nodecimaldot]{babel}  % ponemos tabla en vez de cuadro, sep decimal el punto (Us- notation)
\usepackage[spanish, es-tabla, es-nodecimaldot]{babel}  % ponemos tabla en vez de cuadro, sep decimal el punto (Us- notation)

%\usepackage{ucs}
\usepackage[utf8x]{inputenc} %Paquete para escribir acentos y otros símbolos directamente

\usepackage{pdflscape}
\usepackage{multirow, booktabs,setspace,caption,longtable}
\usepackage{tikz}

% Cambios FV:
% \DeclareCaptionLabelSeparator*{spaced}{\\[2ex]}
% \captionsetup[table]{textfont=it,format=plain,justification=justified,
%   singlelinecheck=false,labelsep=spaced,skip=0pt}
% \captionsetup[figure]{labelsep=period,labelfont=it,justification=justified,
%   singlelinecheck=false,font=doublespacing}

\usepackage[figuresleft]{rotating}
% \usepackage{enumerate}
\usepackage{enumitem}
\usepackage{graphicx}
\usepackage[nottoc]{tocbibind}


\setlength{\parskip}{3mm}

%\usepackage[pdftex,
%            pdfauthor={Your Name},
%            pdftitle={The Title},
%            pdfsubject={The Subject},
%            pdfkeywords={Some Keywords},
%            pdfproducer={Latex with hyperref, or other system},
%            pdfcreator={pdflatex, or other tool}]{hyperref}

\usepackage[
	%paperwidth=17cm,     % como hago para que trome A4 (sin meterle los cms)???
	%paperheight=23.5cm, 
	inner=1.5cm,
	outer=1.5cm,
	top=2.5cm,
	bottom=2cm,
	bindingoffset=1cm]{geometry}

\setlength{\captionmargin}{20pt}

%\usepackage[table, xcdraw]{xcolor}

% % bloques verbarin coloreados segun estilo Rstudio para código R
% % ------------------------------------------------
% \usepackage{color}
% \usepackage{fancyvrb}
% \newcommand{\VerbBar}{|}
% \newcommand{\VERB}{\Verb[commandchars=\\\{\}]}
% \DefineVerbatimEnvironment{Highlighting}{Verbatim}{commandchars=\\\{\}}
% % Add ',fontsize=\small' for more characters per line
% \usepackage{framed}
% \definecolor{shadecolor}{RGB}{248,248,248}
% \newenvironment{Shaded}{\begin{snugshade}}{\end{snugshade}}
% \newcommand{\KeywordTok}[1]{\textcolor[rgb]{0.13,0.29,0.53}{\textbf{{#1}}}}
% \newcommand{\DataTypeTok}[1]{\textcolor[rgb]{0.13,0.29,0.53}{{#1}}}
% \newcommand{\DecValTok}[1]{\textcolor[rgb]{0.00,0.00,0.81}{{#1}}}
% \newcommand{\BaseNTok}[1]{\textcolor[rgb]{0.00,0.00,0.81}{{#1}}}
% \newcommand{\FloatTok}[1]{\textcolor[rgb]{0.00,0.00,0.81}{{#1}}}
% \newcommand{\CharTok}[1]{\textcolor[rgb]{0.31,0.60,0.02}{{#1}}}
% \newcommand{\StringTok}[1]{\textcolor[rgb]{0.31,0.60,0.02}{{#1}}}
% \newcommand{\CommentTok}[1]{\textcolor[rgb]{0.56,0.35,0.01}{\textit{{#1}}}}
% \newcommand{\OtherTok}[1]{\textcolor[rgb]{0.56,0.35,0.01}{{#1}}}
% \newcommand{\AlertTok}[1]{\textcolor[rgb]{0.94,0.16,0.16}{{#1}}}
% \newcommand{\FunctionTok}[1]{\textcolor[rgb]{0.00,0.00,0.00}{{#1}}}
% \newcommand{\RegionMarkerTok}[1]{{#1}}
% \newcommand{\ErrorTok}[1]{\textbf{{#1}}}
% \newcommand{\NormalTok}[1]{{#1}}
% % ------------------------------------------------

\usepackage{pdfpages} % insertar pdf en los anexos
%\usepackage{cite} % para contraer referencias

\usepackage{fancyhdr}
\pagestyle{fancy}

\usepackage{placeins}  % para usar \FloatBarrier
%\usepackage{minitoc}
\usepackage{float}

\usepackage{lipsum}
\usepackage{setspace}

\usepackage{caption}

% BIBLIOGRAFIA (FV):
% \usepackage{biblatex}
% \usepackage[style=apa]{biblatex}
% \addbibresource{ref.bib}

% \usepackage{bibentry}
% \bibliographystyle{apalike}
%\bibliographystyle{unsrt} % igual q plain pero numera en orden de aparición
%\bibliographystyle{apacite}
%\bibliographystyle{plainnat}
% \usepackage{apacite} %para usar APA cite en bibligrafia tb modificar \bibliographystyle{apacite} en su ssitio 
% \DefineBibliographyStrings{spanish}{ andothers = {et\addabbrvspace al\adddot}, and = {y}, }
% \renewcommand{\BOthers}[1]{et al.\hbox{}}
% \addto\captionsspanish{
	%   \renewcommand{\BOthers}[1]{et al.\hbox{}}
	% }
	
%aml para solventar el tema de las citas intex.. 20220405
\usepackage[sort&compress,square,comma,authoryear]{natbib}
% FV para poder incluir full cite con \bibentry:
\usepackage{bibentry}

\usepackage{url}
\usepackage{hyperref}

\nobibliography*


% makes color citations
%\usepackage[colorlinks=true,urlcolor=blue,citecolor=red,linkcolor=red,bookmarks=true]{hyperref}


% \usepackage{ulem} %para tachar frases
