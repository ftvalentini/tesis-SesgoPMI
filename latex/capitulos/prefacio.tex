
\chapter*{Prefacio}

Esta tesis de maestría está basada en los hallazgos presentados en la siguiente publicación, la cual es el resultado de las investigaciones realizadas durante el desarrollo de la tesis:

\bibentry{valentini2023problems}


%En este trabajo emplearemos las siguientes convenciones: 

%\begin{itemize}
%\item 
%Se empleará la tipografía \textit{cursiva} para palabras en otro idioma distinto del castellano y para resaltar nombres como, por ejemplo, los nombres de variables.
%\item 
%Emplearemos la tipografía \texttt{código} para sentencias de código,  nombres de funciones y librerías de \texttt{R}.
%\item 
%Emplearemos el punto como separador decimal, por ejemplo: \texttt{1/2=0.5}, y un espacio simple como separador de millares, por ejemplo: \texttt{123\;345} tal como recomienda la Academia de la Lengua Española en la Ortografía del 2010, página 666 \cite{espanola2010ortografia}. 
%\item 
%Cuando informemos sobre una media, la acompañaremos entre paréntesis de la desviación típica, por ejemplo: \texttt{8.81(1.21)}. 
%\item 
%Si informamos de un porcentaje, entre paréntesis, incluimos la frecuencia absoluta, por ejemplo: \texttt{50.25\%(123)}, o viceversa: \texttt{123(50.25\%)}. 
%\item 
%En las tablas de descriptivos, empleamos la abreviatura \texttt{SD} para \emph{desviación estandar} (del inglés \emph{Standard  Deviation}).
%\item 
%En el texto empleamos la abreviatura \texttt{EPV} para \emph{Enfermedades Prevenibles por Vacunas}.
%\end{itemize}

