
\chapter{Conclusiones} \label{cap:conclusiones}

En los últimos años, la temática de los sesgos en los modelos de aprendizaje automático ha suscitado una gran atención. Aunque numerosos estudios han explorado los sesgos presentes en los modelos, siguen siendo escasas las herramientas diseñadas para medir y analizar directamente los sesgos en los textos. Además, estas herramientas suelen carecer de interpretabilidad. En esta tesis, abordamos estas deficiencias introduciendo y analizando una métrica basada en \emph{Pointwise Mutual Information} (PMI) para medir sesgos en \emph{corpora}. A través de nuestra investigación, hemos destacado las diferencias del enfoque basado en PMI sobre las métricas tradicionales basadas en \emph{word embeddings} estáticos, como SGNS, GloVe y FastText, poniendo especial énfasis en las ventajas.

%A través de nuestra investigación, hemos hecho varias contribuciones importantes y destacado las ventajas del enfoque basado en PMI sobre las métricas tradicionales basadas en \emph{word embeddings} estáticos, como SGNS, GloVe y FastText.

Una de las principales contribuciones de nuestro trabajo es la introducción de la métrica basada en PMI como \textbf{método sencillo, interpretable y computacionalmente efeciente de medir sesgos textuales}. A diferencia de las métricas basadas en \emph{embeddings}, que carecen de transparencia e interpretabilidad, nuestro enfoque ofrece una interpretación clara en términos de coocurrencias de primer orden, y por lo tanto, una comprensión más intuitiva de los sesgos subyacentes presentes en los textos.

Además, introducimos una \textbf{técnica paramétrica para estimar la incertidumbre asociada a las estimaciones} de la métrica basada en PMI. Esta manera de medir la variabilidad permite a los investigadores determinar hasta qué punto los valores medidos pueden atribuirse a fluctuaciones estadísticas. A diferencia de las metodologías tradicionales basadas en el remuestreo, como el \emph{bootstrapping} y las pruebas de permutación utilizadas habitualmente con los \emph{embeddings}, nuestro enfoque paramétrico brinda una estimación más informativa de la verdadera variabilidad en las mediciones de sesgo.

Mediante una serie de experimentos, presentamos \textbf{evidencia empírica que respalda las ventajas del método basado en PMI} frente a las métricas basadas en \emph{embeddings}. Estos resultados demuestran, asimismo, que la métrica basada en PMI muestra asociaciones similares con el juicio humano que las métricas basadas en \emph{embeddings} en determinados escenarios, como los sesgos de género; mientras que en otros casos, como los estereotipos étnicos, PMI y \emph{embeddings} arrojan resultados divergentes. Esta distinción subraya las diferencias fundamentales en las asociaciones semánticas capturadas por los \emph{embeddings} y PMI, respectivamente.

La disponibilidad de herramientas para medir sesgos en los textos es limitada, y la interpretabilidad de estas herramientas es aún más acotada. Nuestro trabajo aborda esta necesidad proporcionando un método transparente e interpretable. Creemos que los estudios centrados en la interpretabilidad y las propiedades estadísticas de las métricas son de suma importancia para el NLP. Al facilitar análisis más transparentes e interpretables, nuestro enfoque puede ayudar a los investigadores del área a hacer análisis más rigurosos de los sesgos potencialmente presentes en textos. En última instancia, nuestro trabajo puede contribuir a mejorar la equidad y la imparcialidad de los sistemas de Inteligencia Artificial, pues permite también \textbf{estudiar y controlar los sesgos de los datos de entrenamiento}.

Además, nuestra contribución es valiosa para los estudios de ciencias sociales computacionales. La métrica basada en PMI puede funcionar como una \textbf{herramienta cuantitativa que complementa a los análisis lingüísticos y sociológicos existentes, más cualitativos, de los sesgos culturales}, mejorando así el conjunto de herramientas analíticas a disposición de los investigadores sociales. Para los estudios en ciencias sociales es particularmente importante no sólo disponer de una métrica transparente para cuantificar estereotipos, sino también de pruebas estadísticas e intervalos de confianza que capten la variabilidad relevante.

Subrayamos la importancia de la interpretabilidad y el rigor estadístico en el desarrollo de herramientas de medición de sesgos en general, y animamos a seguir explorando y perfeccionando estos métodos. 

